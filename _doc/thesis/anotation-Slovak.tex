%%%%%%%%%%%%%%%%%%%%%%%%%%%%%%%%%%%%%%%%%%%%%%%%%%%%%%%%%%%%%%%%%%%%%%%%%%%%%%%%%%%%%%%%
%%
%% Slovak annotation
%%
%%%%%%%%%%%%%%%%%%%%%%%%%%%%%%%%%%%%%%%%%%%%%%%%%%%%%%%%%%%%%%%%%%%%%%%%%%%%%%%%%%%%%%%%
Analýza dát z používateľského správania k zdrojom na webe je v súčasnosti populárna téma, vzhľadom na svoj potenciál zlepšovať služby poskytované návštevníkom webu. Najnovšie prístupy skúmajú aj možnosti aplikácie metód strojového učenia. 
V tejto práci sa zameriavam na možnosti využitia rekurentných neurónových sietí s dlhou krátkodobou pamäťou(LSTM) pri analýze dát z platobných brán pre používanie online spravodajských portálov. Takáto analýza poskytuje náhľad do používateľského správania a z toho vyplývajúce možnosti spätnej väzby voči návštevníkom online spravodajských portálov. Manažment biznisu orientovaného na služby si čoraz viac uvedomuje cenu verného zákazníka na trhu. Je preto nutné odhaliť zákazníka uvažujúceho o prechode ku konkurencii pomocou jeho správania  a pozitívne motivovať jeho vernosť.

