%%%%%%%%%%%%%%%%%%%%%%%%%%%%%%%%%%%%%%%%%%%%%%%%%%%%%%%%%%%%%%%%%%%%%%%%%%%%%%%%%%%%%%%%
%%
%% Slovak annotation
%%
%%%%%%%%%%%%%%%%%%%%%%%%%%%%%%%%%%%%%%%%%%%%%%%%%%%%%%%%%%%%%%%%%%%%%%%%%%%%%%%%%%%%%%%%
Analýza dát z používateľského správania k zdrojom na webe je v súčasnosti populárna téma, vzhľadom na svoj potenciál zlepšovať služby poskytované návštevníkom webu. Najnovšie prístupy skúmajú aj možnosti aplikácie metód strojového učenia. 
Medzi týmito prístupmi si získavajú popularitu hlboké mnohovrstvové samoučiace sa neurónové siete a rôzne architektúry rekurentných neurónových sietí. Využívajú princípy učenia bez učiteľa, pomocou ktorých dokážeme v jednotlivých vzorkách dát identifikovať podstatné črty. V tejto práci sa zameriavam na možnosti využitia rekurentných neurónových sietí s dlhou krátkodobou pamäťou(LSTM) pri analýze dát z platobných brán pre používanie online spravodajských portálov. Takáto analýza poskytuje náhľad do používateľského správania a z toho vyplývajúce možnosti spätnej väzby voči návštevníkom online spravodajských portálov.  

