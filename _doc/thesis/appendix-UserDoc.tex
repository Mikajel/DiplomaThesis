
\chapter{Používateľská dokumentácia}

Pre spustenie programu je nutné mať:

\begin{my_itemize}
	\item{Python verzia 2.7 alebo 3.5}
	\item{Jupyter notebooks}
	\item{Zdrojové súbory programu}
\end{my_itemize}

Program je spustený nasledovnými krokmi:
\begin{my_enumerate}
	\item{Spustenie Jupyter notebooku} - pre spustenie je nutné zadať príkaz 'jupyter-notebook' v konzole. Po štarte Jupyter oznámi na akom porte pracuje. Štandardne sa snaží dostať na port 8888, pokiaľ nie je dostupný, hľadá najbližší voľný port v poradí.
	\item{Otvorenie notebooku s projektom} - Jupyter poskytuje prehliadačové rozhranie pre prácu s notebookmi. Pre otvorenie rozhrania je nutné zadať 'localhost://portnumber' do adresy prehliadača. 'portnumber' reprezentuje číslo prideleného portu, teda štandardne 8888. Následne stačí otvoriť súbor notebooku v prehliadači súborového systému.
	\item{Spustenie vykonatelného kódu} - Pre spustenie jednotlivých buniek stačí kliknúť na bunku a stlačiť Ctrl+Enter. Bunky sú usporiadané v poradí, v akom sa majú spúšťať. Bunky sú číslované podľa poradia spúšťania buniek. Bunka ktorá stále pracuje, má miesto čísla hviezdičku.
	
	Pre prepnutie datasetu na menší alebo väčší súbor stačí prepísať názov súboru v prvej bunke. Relatívne cesty sú nastavené tak, aby sa do nich nemuselo zasahovať. Dostupné súbory sú v zložke 'data'.
	
	Kód vypisuje informácie pod bunku, v ktorej aktuálne beží. Mnoho metód má definovaných parameter 'info', ktorého nastavenie na 'True' alebo 'False'  ovplyvňuje množstvo výpisu.
\end{my_enumerate}
