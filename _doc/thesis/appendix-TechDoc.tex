
\chapter{Technická dokumentácia}
\label{technicka_dokumentacia}
%\section{Implementation}
\section{Implementácia}

Program je implementovaný v jazyku Python. Je upravený pre prácu s obomi verziami Python (2.X aj 3.X). Hlavný program je implementovaný v Python Jupyter Notebooku.

\paragraph{Modul DataHandler}
obsauje metódy určené na prácu s dátami. Nachádzajú sa tu metódy na:
\begin{my_itemize}
	\item{načítanie dát}
	\item{oversample a undersample}
	\item{časové údaje}
	\item{výpisy}
	\item{počítanie úspešnosti učenia}
	\item{normalizácia}
\end{my_itemize} 

\paragraph{Modul ModelInput}
obsahuje definície tried pre dátové objekty reprezentujúce dáta využívané v predspracovaní. Obsahuje tiež metódy, ktoré súvisia s predspracovaním dát, agregáciou a zmenou na vstupné vektory.

Triedy:
\begin{my_itemize}
	\item{Session} - Inštancie tejto triedy predstavujú agregované dáta z klikov patriacich pod unikátnu používateľskú session.
	\item{SessionObject} - Inštancie sú predspracovaním dát, obsahujú údaje, ktoré sú vkladané na vstup do neurónovej siete po zakódovaní na vstupné vektory.
\end{my_itemize}

