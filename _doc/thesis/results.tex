\newpage

\chapter{Výsledky}

\section{Trénovanie}

Trénovanie prebiehalo za použitia optimalizačnej stratégie ADAM. Počiatočná rýchlosť učenia je 0.001. Počet neurónov skrytej vrstvy je 128.\newline
Pracujeme s podmnožinou o veľkosti 100 tisíc klikov, z ktorej je vygenerovaných 20 tisíc vstupných vektorov pre tréningovú množinu. Validačná a testovacia množina obsahujú 2,6 tisíc vektorov, tj. 10 \% datasetu pre každú. Po vyvážení tried prebieha učenie na vzorke 18 tisíc vstupných vektorov pre triedu nákupných aj nenákupných sessions. Spolu je teda sieť učená na datasete 37 tisíc vzoriek. Vzorky sú pred vstupom zamiešané a rozdelené do úsekov. Po každom úseku je evaluovaná sieť voči validačnej množine.\newline
Tréning je zastavovaný, keď validačná množina prestáva zaznamenávať zlepšovanie oproti predchádzajúcej epoche. Keďže je štatisticky možné aby sa objavilo jedno zhoršenie počas tréningu, sieť je zastavovaná ak na validačnej množine nedojde k zlepšeniu väčšiemu ako 0.25 \% 3x po sebe. 

\section{Testovanie}
\label{testing}

Popri samotnej validačnej množine sú vyhodnocované aj metriky tried klasifikácie:

\begin{my_itemize}
	\myitem{True Positive}	
	\myitem{True Negative}
	\myitem{False Positive}
	\myitem{False Negative}
\end{my_itemize}

