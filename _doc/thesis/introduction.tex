\newpage
\chapter{Úvod}

Ľudský mozog je zložitý a ešte v dnešných dobách z veľkej časti nepochopený orgán. Od prvého okamihu je trénovaný riešiť problémy, ktorých formálna špecifikácia presahuje naše možnosti. Snahou neurónových sietí je napodobniť takéto schopnosti a možnosti simuláciou architektúry mozgu. \newline
Jednou z možných aplikácií takýchto schopností je prenikanie do mysle zákazníka na trhu služieb. Online biznis má prostriedky, ktoré mu dovoľujú zhromaždiť obrovské množstvo dát o aktivite svojich zákazníkov. Na získanie pridanej hodnoty z týchto dát je však nutné identifikovať a pochopiť vzory, ktoré sa v týchto dátach nachádzajú. Takáto úloha je pre neurónové siete adekvátnou výzvou. 
\newpage

\section{Použité skratky}
\label{uvod_pouzite_skratky}

\begin{my_itemize}
\item \textbf{NN, ANN} - Neurónová sieť (Neural Network, Artificial Neural Network)
\item \textbf{FNN} - Dopredná neurónová sieť (Feedforward Neural Network)
\item \textbf{RNN} - Rekurentná neurónová sieť (Recurrent Neural Network)
\item \textbf{LSTM} -Dlhá krátkodobá pamäť (Long Short-Term memory)
\item \textbf{CRM} - Manažment vzťahov so zákazníkmi (Customer Relationship Management)
\item \textbf{VAW} - Web s pridanou hodnotou (Value Added Web)
\item \textbf{TINA} - Informačná sieťová architektúra v telekomunikáciach (Telecommunication Information Network Architecture)
\item \textbf{IP} - Internetový protokol (Internet Protocol)
\item \textbf{CEC} - Konštantné cyklenie chyby (Constant Error Carousel)
\item \textbf{AUC} - Oblasť pod krivkou (Area under curve)
\end{my_itemize}
